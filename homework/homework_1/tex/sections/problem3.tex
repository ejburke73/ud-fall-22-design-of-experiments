\documentclass[../main.tex]{subfiles}

\begin{document}

\problem{2.28}

The following are the burning times (in minutes) of chemical flares of two different formulations.
The design engineers are interested in both the mean and variance of the burning times.

\begin{table}[ht]
    \centering
    \begin{tabular}{cccc}
    \toprule
    \multicolumn{2}{c}{\textbf{Type 1}} & \multicolumn{2}{c}{\textbf{Type 2}} \\ \midrule
    65     & 82    & 64    & 56    \\ 
    81     & 67    & 71    & 69    \\ 
    57     & 59    & 83    & 74    \\ 
    66     & 75    & 59    & 82    \\ 
    82     & 70    & 65    & 79    \\ \toprule
    \end{tabular}
\end{table}

\begin{enumerate}[label = (\alph*)]
    \item\label{28A}
        Test the hypothesis that the two variances are equal.
        Use \(\alpha = 0.05\)

        \givens{}
        Data in table in minutes.\\
        \(\alpha = 0.05\)

        \assumptions{}
        The data is normally distributed.

        \solution{}
        Claim: The variances of burning times of both formulations of chemical flares are equal.
        \[\sigma_1 = \sigma_2\]

        Hypotheses:

        \[H_0' : \sigma_1 = \sigma_2\]
        \[H_1' : \sigma_1 \ne \sigma_2\]

        The test statistic for comparing variances of two different samples is $F$ and is given by

        \[
            F = \frac{S_1^2}{S_2^2}    
        \]

        Using the Ti-84's 2Samp-FTest function, we obtain the value of our test statistic $F$, as well as its $P$-value.

        \[F = 0.9782\]
        \[P\textrm{-value} = 0.9744\]

        This value is sufficiently large to move forward assuming that \(\sigma_1=\sigma_2\). 
        In other words, at \(\alpha = 0.05\), we fail to reject \(H_0' : \sigma_1 = \sigma_2\).
        We cannot prove that there is a statistically significant difference between the variances at \(\alpha = 0.05\).
    \item 
        Using the results from~\ref{28A}, test the hypothesis that the mean burning times are equal.
        Use \(\alpha = 0.05\).
        What is the $P$-value for this test?

        \givens{}
        \(\alpha = 0.05\)

        \assumptions{}
        From part \ref{28A}, we need to use a pooled t-test for comparison of means between the two samples.
        The data is normally distributed.

        \solution{}

        Claim: The mean burning times are equal.
        \[\mu_1 = \mu_2\]

        Hypotheses:
        \[H_0 : \mu_1 = \mu_2\]
        \[H_1 : \mu_1 \ne \mu_2\]

        Test Statistic:
        \[
            t_0 = \frac{\bar{y_1}-\bar{y_2}}{S_p \sqrt{\frac{1}{n_1} + \frac{1}{n_2}}}
        \]

        Using the Ti-84's 2-SampTTest function to perform a pooled two-sample t-test on the given data, we obtain the value of the test statistic and its $P$-value.

        \[t_0 = 0.0480\]
        \[P\textrm{-value} = 0.9622\]
        \[0.9622 > 0.05 \rightarrow \textrm{Fail to reject} \ H_0\]

        The $P$-value is large in comparison to \(\alpha = 0.05\), therefore we fail to reject \(H_0 : \mu_1 = \mu_2\).
        We cannot prove that there is a statistically meaningful difference between the mean burning times at \(\alpha = 0.05\).
        
\end{enumerate}

\end{document}