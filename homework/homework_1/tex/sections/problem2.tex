\documentclass[../main.tex]{subfiles}

\begin{document}

\problem{2.24}

The time to repair an electronic instrument is a normally distributed random variable measured in hours.
The repair times for 16 such instruments chosen at random are as follows:

\begin{table}[ht]
    \centering
    \begin{tabular}{llll}
    \toprule
    \multicolumn{4}{c}{\textbf{Hours}} \\ \midrule
    159     & 280    & 101    & 212    \\ 
    224     & 379    & 179    & 264    \\ 
    222     & 362    & 168    & 250    \\ 
    149     & 260    & 485    & 170    \\ \toprule
    \end{tabular}
\end{table}

\begin{enumerate}[label=(\alph*)]
    \item\label{24A}
    You wish to know if the mean repair time exceeds 225 hours.
    Set up appropriate hypotheses for investigating this issue.

    \givens{}
    \(n = 16\)\\
    Time is normally distributed and random.
    
    \solution{}
    Claim: The mean repair time exceeds 225 hours.
    \[\mu > 225\]

    Hypotheses:
    \[H_0 : \mu = 225\] 
    \[H_1 : \mu > 225\]

    \item
    Test the hypotheses formulated in part~\ref{24A}.
    What are your conclusions?
    Use \(\alpha = 0.05\).

    \givens{}
    \(\alpha = 0.05\)

    \solution{}
    For a simple hypothesis test of one variable without a known population variance, we use a t-test.
    Because normality is assumed in the givens, we do not need to test for it. 
    Our test statistic, \(t\), is defined as:

    \[
        t = \frac{\bar{y}-\mu}{S/\sqrt{n}} \, ,
    \]

    where \(\bar{y}\) is the sample mean, \(\mu\) is the population mean, \(S\) is the sample variance, and \(n\) is the size of the sample.

    The critical value is the value of the test statistic beyond which we reject the null hypothesis. 
    The C.V. for a one-tailed test is given by

    \[
        C.V. = t_{\alpha,n-1}
    \]

    The C.V. can be found using the Ti-84's invT() function, which yields a $t$-value for a given area and number of degrees of freedom.

    \[
        {t_{0.95,15} = 1.7530}     
    \]    
    
    Using the T-Test function on the Ti-84 with the data from the table yields the value of the test statistic, $t$.

    \[{t = 0.6685}\]

    Our test statistic, $t$, is smaller than our critical value, therefore we fail to reject the null hypothesis $H_0$ at $\alpha = 0.05$.

    \[
        \boxed{0.6685 < 1.7530 \rightarrow \textrm{Fail to reject} \ H_0}    
    \]    

    This indicates that we cannot prove that the mean repair time is not 225 hours at \(\alpha = 0.05\).

    \item
    Find the $P$-value for the test.

    \solution{}
    Using the T-test function on the Ti-84 with the data from the table yields the $P$-value.

    \[\boxed{P\textrm{-value} = 0.2570}\]
    
    As before, this value is large compared to \(\alpha = 0.05\), therefore we fail to reject \(H_0\). 

    \item
    Construct a 95 percent confidence interval on mean repair time.

    \solution{}

    A confidence interval on a mean for a t-distribution is defined as

    \[ 
        P
        \left({
            \bar{y} - t_{\alpha/2,n-1} S/\sqrt{n} \le \mu \le \bar{y} + t_{\alpha/2,n-1} S/\sqrt{n}
        }\right)  
        = 1 - \alpha 
    \]

    Using the Ti-84's TInterval function yields the 95 \% confidence interval on mean repair time.

    \[
        \boxed{\left(188.89 \le \mu \le 294.11\right)}
    \]

    This result indicates we are 95\% confident that the true value of mean repair time falls between 188.89 hours and 294.11 hours. 
    
\end{enumerate}

\end{document}