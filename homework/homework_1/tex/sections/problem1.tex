\documentclass[../main.tex]{subfiles}

\begin{document}

\problem{2.7}

Suppose that we are testing \(H_0 : \mu_1 = \mu_2\) versus \(H_1 : \mu_1 > \mu_2\) where the two sample sizes are \(n_1 = n_2 = 10\).
Both sample variances are unknown but assumed equal. Find bounds on the $P$-value for the following observed values of the test statistic.

\begin{enumerate}[label = (\alph*)]
    \item \(t_0 = 2.31\)
    \item \(t_0 = 3.60\)
    \item \(t_0 = 1.95\)
    \item \(t_0 = 2.19\)
\end{enumerate}

\givens{}
\(n_1 = n_2 = 10\) \\
Sample variances are unknown but equal.

\assumptions{}
For all values of the test statistic, we use a two-sample t-test because the \textit{population variances} are unknown.
This is a right-tailed t-test due to the alternative hypothesis, \(H_1 : \mu_1 > \mu_2\).
We use a pooled t-test because the sample variances are assumed equal. 
The t-test has \(n_1+n_2-2 = 18\) degrees of freedom.

\solution{}

Claim: Population mean 1, \(\mu_1\), is equal to population mean 2, \(\mu_2\).
\[\mu_1 = \mu_2\]

Hypotheses:
\[H_0 : \mu_1 = \mu_2\] 
\[H_1 : \mu_1 > \mu_2\]
   
 Because the sample variances are assumed equal, we do not perform an F-test, and move directly to a pooled t-test for our sample.
 The test statistic for a pooled t-test is given as

 \[
    t_0 = \frac{\bar{y_1}-\bar{y_2}}{S_p \sqrt{\frac{1}{n_1} + \frac{1}{n_2}}}   \, ,
 \]

where \(S_p\) is defined as

\[
    S_p = \sqrt{
        \frac{(n_1-1)S_1^2 + (n_2-1)S_2^2}{n_1+n_2-2} \, ,
    }  
\]

and \(S_1 \ \textrm{and} \ S_2\) are the sample variances.

\par

Given the values of the test statistic, \(t_o\), we use the tcdf() function of the Ti-84 calculator with \(n_1 + n_2 -2\) degrees of freedom to generate a $P$-value.
The tcdf() function's inputs are a lower bound, an upper bound, and the number of degrees of freedom. In our case, the lower bound will be our test statistic, \(t_o\) and the upper bound will be some very large number, in this case, \(E99\) to approximate the infinite end of the t-distribution.
\begin{enumerate}[label = (\alph*)]

    \item \(t_0 = 2.31\) \\
        tcdf(2.31,E99,18)\\
        $P$-value = 0.01647

    \item \(t_0 = 3.60\)\\
        tcdf(3.60,E99,18)\\
        $P$-value = 0.0010

    \item \(t_0 = 1.95\)\\
        tcdf(1.95,E99,18)\\
        $P$-value = 0.0335

    \item \(t_0 = 2.19\)\\
        tcdf(2.19,E99,18)\\
        $P$-value = 0.0210
        
    Given the observed test-statistics, we can bound the $P$-value on:
    
    \[\left({0.0010 \le P \le 0.0335}\right)\]
\end{enumerate}


\end{document}