\documentclass[../main.tex]{subfiles}

\begin{document}

\problem{2.31}

Photoresist is a light-sensitive material applied to semiconductor wafers so that the circuitr pattern can be imaged on to the wafer.
After application, the coated wafers are baked to remove the solvent in the photoresist mixture and to harden the resist.
Here are measurements of photoresist thickness (in kA) for eight wafers baked at two different temperatures.
Assume that all of the runs were made in random order.

\begin{table}[ht]
    \centering
    \begin{tabular}{cc}
    \toprule
    \textbf{95 \textdegree C}  & \textbf{100 \textdegree C}\\ \midrule
    11.176  & 5.263 \\
    7.089   & 6.748 \\
    8.097   & 7.461 \\ 
    11.739  & 7.015 \\ 
    11.291  & 8.133 \\
    10.759  & 7.418 \\
    6.467   & 3.772 \\ 
    8.315   & 8.963 \\ \toprule
    \end{tabular}
\end{table}

\begin{enumerate}[label = (\alph*)]
    \item\label{31A}
        Is there evidence to support the claim that the higher baking temperature results in wafers with a lower mean photoresist thickness?
        Use \(\alpha = 0.05\).

        \givens{}
        \(\alpha = 0.05\)

        \assumptions{}
        Runs were made in random order.

        \solution{}

        Claim: Higher baking temperatures result in wafers with a lower mean photoresist thickness.
        \[\mu_1 > \mu_2\]

        Hypotheses:
        \[H_0 : \mu_1 \le \mu_2\]
        \[H_1 : \mu_1 > \mu_2\]

        In order to test these hypotheses, we will use a right-tailed two-sample t-test.
        Before doing so, we need to test if we need to use a pooled test or not. 

        Claim: 
        \[\sigma_1 = \sigma_2\]
        
        Hypotheses:
        \[H_0' : \sigma_1 = \sigma_2\]
        \[H_1' : \sigma_1 \ne \sigma_2\]
        
        The test statistic for comparing variances of two different samples is $F$ and is given by

        \[
            F = \frac{S_1^2}{S_2^2}    
        \]

        Using the Ti-84's 2Samp-FTest function, we obtain the value of our test statistic $F$, as well as its $P$-value.

        \[F = 1.6381\]
        \[P\textrm{-value} = 0.5306\]

        This value is sufficiently large to move forward assuming that \(\sigma_1=\sigma_2\). 
        In other words, at \(\alpha = 0.05\), we fail to reject \(H_0' : \sigma_1 = \sigma_2\).
        We cannot prove that there is a statistically significant difference between the variances at \(\alpha = 0.05\).

        We can now perform a pooled two-sample t-test.

        Test Statistic:
        \[
            t_0 = \frac{\bar{y_1}-\bar{y_2}}{S_p \sqrt{\frac{1}{n_1} + \frac{1}{n_2}}}
        \]

        Using the Ti-84's 2-SampTTest function to perform a right-tailed pooled two-sample t-test on the given data, we obtain the value of the test statistic and its $P$-value.

        \[t_0 = 2.6751\]
        \[P\textrm{-value} = 0.0091\]
        \[0.0091 < 0.05 \rightarrow \textrm{Reject} \ H_0\]

        The $P$-value is small in comparison to \(\alpha = 0.05\), therefore we reject \(H_0 : \mu_1 \le \mu_2\).
        At \(\alpha = 0.05\) we have shown a statistically significant difference in means indicating that the mean photoresist thickness is lower when the baking temperature is higher.
        
    \item
        What is the $P$-value for the test conducted in part~\ref{31A}?

        \solution{}

        The $P$-value is 0.0091.

    \item
        Find a 95 percent confidence interval on the difference in means.
        Provide a practical interpretation of this interval.

        \solution{}

        A (1-$\alpha$)$\times100$\% confidence interval on the difference in means for a t-distribution is given by 

        \[ 
            \left({
                (\bar{Y}_1-\bar{Y}_2) - t_{\alpha/2,n_1+n_2-2} S_p \sqrt{\frac{1}{n_1} + \frac{1}{n_2}} \le \mu_1-\mu_2 \le (\bar{Y}_1-\bar{Y}_2) + t_{\alpha/2,n_1+n_2-2} S_p \sqrt{\frac{1}{n_1} + \frac{1}{n_2}}
            }\right)  
        \]

        Using the Ti-84's 2-SampTInterval function yields the 95 \% confidence interval on the difference in mean photoresist thickness.

        \[
            \boxed{\left(0.49957  \le \mu_1-\mu_2 \le 4.5405 \right)}
        \]

        This result indicates we are 95\% confident that the true value of the difference in mean thicnkess falls between 0.49957 and 4.5405 kA.
        

\end{enumerate}

\end{document}